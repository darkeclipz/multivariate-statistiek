
% Default to the notebook output style

    


% Inherit from the specified cell style.




    
\documentclass[11pt]{article}

    
    
    \usepackage[T1]{fontenc}
    % Nicer default font (+ math font) than Computer Modern for most use cases
    \usepackage{mathpazo}

    % Basic figure setup, for now with no caption control since it's done
    % automatically by Pandoc (which extracts ![](path) syntax from Markdown).
    \usepackage{graphicx}
    % We will generate all images so they have a width \maxwidth. This means
    % that they will get their normal width if they fit onto the page, but
    % are scaled down if they would overflow the margins.
    \makeatletter
    \def\maxwidth{\ifdim\Gin@nat@width>\linewidth\linewidth
    \else\Gin@nat@width\fi}
    \makeatother
    \let\Oldincludegraphics\includegraphics
    % Set max figure width to be 80% of text width, for now hardcoded.
    \renewcommand{\includegraphics}[1]{\Oldincludegraphics[width=.8\maxwidth]{#1}}
    % Ensure that by default, figures have no caption (until we provide a
    % proper Figure object with a Caption API and a way to capture that
    % in the conversion process - todo).
    \usepackage{caption}
    \DeclareCaptionLabelFormat{nolabel}{}
    \captionsetup{labelformat=nolabel}

    \usepackage{adjustbox} % Used to constrain images to a maximum size 
    \usepackage{xcolor} % Allow colors to be defined
    \usepackage{enumerate} % Needed for markdown enumerations to work
    \usepackage{geometry} % Used to adjust the document margins
    \usepackage{amsmath} % Equations
    \usepackage{amssymb} % Equations
    \usepackage{textcomp} % defines textquotesingle
    % Hack from http://tex.stackexchange.com/a/47451/13684:
    \AtBeginDocument{%
        \def\PYZsq{\textquotesingle}% Upright quotes in Pygmentized code
    }
    \usepackage{upquote} % Upright quotes for verbatim code
    \usepackage{eurosym} % defines \euro
    \usepackage[mathletters]{ucs} % Extended unicode (utf-8) support
    \usepackage[utf8x]{inputenc} % Allow utf-8 characters in the tex document
    \usepackage{fancyvrb} % verbatim replacement that allows latex
    \usepackage{grffile} % extends the file name processing of package graphics 
                         % to support a larger range 
    % The hyperref package gives us a pdf with properly built
    % internal navigation ('pdf bookmarks' for the table of contents,
    % internal cross-reference links, web links for URLs, etc.)
    \usepackage{hyperref}
    \usepackage{longtable} % longtable support required by pandoc >1.10
    \usepackage{booktabs}  % table support for pandoc > 1.12.2
    \usepackage[inline]{enumitem} % IRkernel/repr support (it uses the enumerate* environment)
    \usepackage[normalem]{ulem} % ulem is needed to support strikethroughs (\sout)
                                % normalem makes italics be italics, not underlines
    \usepackage{mathrsfs}
    

    
    
    % Colors for the hyperref package
    \definecolor{urlcolor}{rgb}{0,.145,.698}
    \definecolor{linkcolor}{rgb}{.71,0.21,0.01}
    \definecolor{citecolor}{rgb}{.12,.54,.11}

    % ANSI colors
    \definecolor{ansi-black}{HTML}{3E424D}
    \definecolor{ansi-black-intense}{HTML}{282C36}
    \definecolor{ansi-red}{HTML}{E75C58}
    \definecolor{ansi-red-intense}{HTML}{B22B31}
    \definecolor{ansi-green}{HTML}{00A250}
    \definecolor{ansi-green-intense}{HTML}{007427}
    \definecolor{ansi-yellow}{HTML}{DDB62B}
    \definecolor{ansi-yellow-intense}{HTML}{B27D12}
    \definecolor{ansi-blue}{HTML}{208FFB}
    \definecolor{ansi-blue-intense}{HTML}{0065CA}
    \definecolor{ansi-magenta}{HTML}{D160C4}
    \definecolor{ansi-magenta-intense}{HTML}{A03196}
    \definecolor{ansi-cyan}{HTML}{60C6C8}
    \definecolor{ansi-cyan-intense}{HTML}{258F8F}
    \definecolor{ansi-white}{HTML}{C5C1B4}
    \definecolor{ansi-white-intense}{HTML}{A1A6B2}
    \definecolor{ansi-default-inverse-fg}{HTML}{FFFFFF}
    \definecolor{ansi-default-inverse-bg}{HTML}{000000}

    % commands and environments needed by pandoc snippets
    % extracted from the output of `pandoc -s`
    \providecommand{\tightlist}{%
      \setlength{\itemsep}{0pt}\setlength{\parskip}{0pt}}
    \DefineVerbatimEnvironment{Highlighting}{Verbatim}{commandchars=\\\{\}}
    % Add ',fontsize=\small' for more characters per line
    \newenvironment{Shaded}{}{}
    \newcommand{\KeywordTok}[1]{\textcolor[rgb]{0.00,0.44,0.13}{\textbf{{#1}}}}
    \newcommand{\DataTypeTok}[1]{\textcolor[rgb]{0.56,0.13,0.00}{{#1}}}
    \newcommand{\DecValTok}[1]{\textcolor[rgb]{0.25,0.63,0.44}{{#1}}}
    \newcommand{\BaseNTok}[1]{\textcolor[rgb]{0.25,0.63,0.44}{{#1}}}
    \newcommand{\FloatTok}[1]{\textcolor[rgb]{0.25,0.63,0.44}{{#1}}}
    \newcommand{\CharTok}[1]{\textcolor[rgb]{0.25,0.44,0.63}{{#1}}}
    \newcommand{\StringTok}[1]{\textcolor[rgb]{0.25,0.44,0.63}{{#1}}}
    \newcommand{\CommentTok}[1]{\textcolor[rgb]{0.38,0.63,0.69}{\textit{{#1}}}}
    \newcommand{\OtherTok}[1]{\textcolor[rgb]{0.00,0.44,0.13}{{#1}}}
    \newcommand{\AlertTok}[1]{\textcolor[rgb]{1.00,0.00,0.00}{\textbf{{#1}}}}
    \newcommand{\FunctionTok}[1]{\textcolor[rgb]{0.02,0.16,0.49}{{#1}}}
    \newcommand{\RegionMarkerTok}[1]{{#1}}
    \newcommand{\ErrorTok}[1]{\textcolor[rgb]{1.00,0.00,0.00}{\textbf{{#1}}}}
    \newcommand{\NormalTok}[1]{{#1}}
    
    % Additional commands for more recent versions of Pandoc
    \newcommand{\ConstantTok}[1]{\textcolor[rgb]{0.53,0.00,0.00}{{#1}}}
    \newcommand{\SpecialCharTok}[1]{\textcolor[rgb]{0.25,0.44,0.63}{{#1}}}
    \newcommand{\VerbatimStringTok}[1]{\textcolor[rgb]{0.25,0.44,0.63}{{#1}}}
    \newcommand{\SpecialStringTok}[1]{\textcolor[rgb]{0.73,0.40,0.53}{{#1}}}
    \newcommand{\ImportTok}[1]{{#1}}
    \newcommand{\DocumentationTok}[1]{\textcolor[rgb]{0.73,0.13,0.13}{\textit{{#1}}}}
    \newcommand{\AnnotationTok}[1]{\textcolor[rgb]{0.38,0.63,0.69}{\textbf{\textit{{#1}}}}}
    \newcommand{\CommentVarTok}[1]{\textcolor[rgb]{0.38,0.63,0.69}{\textbf{\textit{{#1}}}}}
    \newcommand{\VariableTok}[1]{\textcolor[rgb]{0.10,0.09,0.49}{{#1}}}
    \newcommand{\ControlFlowTok}[1]{\textcolor[rgb]{0.00,0.44,0.13}{\textbf{{#1}}}}
    \newcommand{\OperatorTok}[1]{\textcolor[rgb]{0.40,0.40,0.40}{{#1}}}
    \newcommand{\BuiltInTok}[1]{{#1}}
    \newcommand{\ExtensionTok}[1]{{#1}}
    \newcommand{\PreprocessorTok}[1]{\textcolor[rgb]{0.74,0.48,0.00}{{#1}}}
    \newcommand{\AttributeTok}[1]{\textcolor[rgb]{0.49,0.56,0.16}{{#1}}}
    \newcommand{\InformationTok}[1]{\textcolor[rgb]{0.38,0.63,0.69}{\textbf{\textit{{#1}}}}}
    \newcommand{\WarningTok}[1]{\textcolor[rgb]{0.38,0.63,0.69}{\textbf{\textit{{#1}}}}}
    
    
    % Define a nice break command that doesn't care if a line doesn't already
    % exist.
    \def\br{\hspace*{\fill} \\* }
    % Math Jax compatibility definitions
    \def\gt{>}
    \def\lt{<}
    \let\Oldtex\TeX
    \let\Oldlatex\LaTeX
    \renewcommand{\TeX}{\textrm{\Oldtex}}
    \renewcommand{\LaTeX}{\textrm{\Oldlatex}}
    % Document parameters
    % Document title
    \title{Principele componenten analyse (PCA): \\ Heart disease dataset}
    \author{Rosa de Haan (rosa.de.haan@student.nhlstenden.com)\\ Lars Rotgers (lars.rotgers@student.nhlstenden.com)}
    \date{4 juni 2019}
    
    
    
    
    

    % Pygments definitions
    
\makeatletter
\def\PY@reset{\let\PY@it=\relax \let\PY@bf=\relax%
    \let\PY@ul=\relax \let\PY@tc=\relax%
    \let\PY@bc=\relax \let\PY@ff=\relax}
\def\PY@tok#1{\csname PY@tok@#1\endcsname}
\def\PY@toks#1+{\ifx\relax#1\empty\else%
    \PY@tok{#1}\expandafter\PY@toks\fi}
\def\PY@do#1{\PY@bc{\PY@tc{\PY@ul{%
    \PY@it{\PY@bf{\PY@ff{#1}}}}}}}
\def\PY#1#2{\PY@reset\PY@toks#1+\relax+\PY@do{#2}}

\expandafter\def\csname PY@tok@w\endcsname{\def\PY@tc##1{\textcolor[rgb]{0.73,0.73,0.73}{##1}}}
\expandafter\def\csname PY@tok@c\endcsname{\let\PY@it=\textit\def\PY@tc##1{\textcolor[rgb]{0.25,0.50,0.50}{##1}}}
\expandafter\def\csname PY@tok@cp\endcsname{\def\PY@tc##1{\textcolor[rgb]{0.74,0.48,0.00}{##1}}}
\expandafter\def\csname PY@tok@k\endcsname{\let\PY@bf=\textbf\def\PY@tc##1{\textcolor[rgb]{0.00,0.50,0.00}{##1}}}
\expandafter\def\csname PY@tok@kp\endcsname{\def\PY@tc##1{\textcolor[rgb]{0.00,0.50,0.00}{##1}}}
\expandafter\def\csname PY@tok@kt\endcsname{\def\PY@tc##1{\textcolor[rgb]{0.69,0.00,0.25}{##1}}}
\expandafter\def\csname PY@tok@o\endcsname{\def\PY@tc##1{\textcolor[rgb]{0.40,0.40,0.40}{##1}}}
\expandafter\def\csname PY@tok@ow\endcsname{\let\PY@bf=\textbf\def\PY@tc##1{\textcolor[rgb]{0.67,0.13,1.00}{##1}}}
\expandafter\def\csname PY@tok@nb\endcsname{\def\PY@tc##1{\textcolor[rgb]{0.00,0.50,0.00}{##1}}}
\expandafter\def\csname PY@tok@nf\endcsname{\def\PY@tc##1{\textcolor[rgb]{0.00,0.00,1.00}{##1}}}
\expandafter\def\csname PY@tok@nc\endcsname{\let\PY@bf=\textbf\def\PY@tc##1{\textcolor[rgb]{0.00,0.00,1.00}{##1}}}
\expandafter\def\csname PY@tok@nn\endcsname{\let\PY@bf=\textbf\def\PY@tc##1{\textcolor[rgb]{0.00,0.00,1.00}{##1}}}
\expandafter\def\csname PY@tok@ne\endcsname{\let\PY@bf=\textbf\def\PY@tc##1{\textcolor[rgb]{0.82,0.25,0.23}{##1}}}
\expandafter\def\csname PY@tok@nv\endcsname{\def\PY@tc##1{\textcolor[rgb]{0.10,0.09,0.49}{##1}}}
\expandafter\def\csname PY@tok@no\endcsname{\def\PY@tc##1{\textcolor[rgb]{0.53,0.00,0.00}{##1}}}
\expandafter\def\csname PY@tok@nl\endcsname{\def\PY@tc##1{\textcolor[rgb]{0.63,0.63,0.00}{##1}}}
\expandafter\def\csname PY@tok@ni\endcsname{\let\PY@bf=\textbf\def\PY@tc##1{\textcolor[rgb]{0.60,0.60,0.60}{##1}}}
\expandafter\def\csname PY@tok@na\endcsname{\def\PY@tc##1{\textcolor[rgb]{0.49,0.56,0.16}{##1}}}
\expandafter\def\csname PY@tok@nt\endcsname{\let\PY@bf=\textbf\def\PY@tc##1{\textcolor[rgb]{0.00,0.50,0.00}{##1}}}
\expandafter\def\csname PY@tok@nd\endcsname{\def\PY@tc##1{\textcolor[rgb]{0.67,0.13,1.00}{##1}}}
\expandafter\def\csname PY@tok@s\endcsname{\def\PY@tc##1{\textcolor[rgb]{0.73,0.13,0.13}{##1}}}
\expandafter\def\csname PY@tok@sd\endcsname{\let\PY@it=\textit\def\PY@tc##1{\textcolor[rgb]{0.73,0.13,0.13}{##1}}}
\expandafter\def\csname PY@tok@si\endcsname{\let\PY@bf=\textbf\def\PY@tc##1{\textcolor[rgb]{0.73,0.40,0.53}{##1}}}
\expandafter\def\csname PY@tok@se\endcsname{\let\PY@bf=\textbf\def\PY@tc##1{\textcolor[rgb]{0.73,0.40,0.13}{##1}}}
\expandafter\def\csname PY@tok@sr\endcsname{\def\PY@tc##1{\textcolor[rgb]{0.73,0.40,0.53}{##1}}}
\expandafter\def\csname PY@tok@ss\endcsname{\def\PY@tc##1{\textcolor[rgb]{0.10,0.09,0.49}{##1}}}
\expandafter\def\csname PY@tok@sx\endcsname{\def\PY@tc##1{\textcolor[rgb]{0.00,0.50,0.00}{##1}}}
\expandafter\def\csname PY@tok@m\endcsname{\def\PY@tc##1{\textcolor[rgb]{0.40,0.40,0.40}{##1}}}
\expandafter\def\csname PY@tok@gh\endcsname{\let\PY@bf=\textbf\def\PY@tc##1{\textcolor[rgb]{0.00,0.00,0.50}{##1}}}
\expandafter\def\csname PY@tok@gu\endcsname{\let\PY@bf=\textbf\def\PY@tc##1{\textcolor[rgb]{0.50,0.00,0.50}{##1}}}
\expandafter\def\csname PY@tok@gd\endcsname{\def\PY@tc##1{\textcolor[rgb]{0.63,0.00,0.00}{##1}}}
\expandafter\def\csname PY@tok@gi\endcsname{\def\PY@tc##1{\textcolor[rgb]{0.00,0.63,0.00}{##1}}}
\expandafter\def\csname PY@tok@gr\endcsname{\def\PY@tc##1{\textcolor[rgb]{1.00,0.00,0.00}{##1}}}
\expandafter\def\csname PY@tok@ge\endcsname{\let\PY@it=\textit}
\expandafter\def\csname PY@tok@gs\endcsname{\let\PY@bf=\textbf}
\expandafter\def\csname PY@tok@gp\endcsname{\let\PY@bf=\textbf\def\PY@tc##1{\textcolor[rgb]{0.00,0.00,0.50}{##1}}}
\expandafter\def\csname PY@tok@go\endcsname{\def\PY@tc##1{\textcolor[rgb]{0.53,0.53,0.53}{##1}}}
\expandafter\def\csname PY@tok@gt\endcsname{\def\PY@tc##1{\textcolor[rgb]{0.00,0.27,0.87}{##1}}}
\expandafter\def\csname PY@tok@err\endcsname{\def\PY@bc##1{\setlength{\fboxsep}{0pt}\fcolorbox[rgb]{1.00,0.00,0.00}{1,1,1}{\strut ##1}}}
\expandafter\def\csname PY@tok@kc\endcsname{\let\PY@bf=\textbf\def\PY@tc##1{\textcolor[rgb]{0.00,0.50,0.00}{##1}}}
\expandafter\def\csname PY@tok@kd\endcsname{\let\PY@bf=\textbf\def\PY@tc##1{\textcolor[rgb]{0.00,0.50,0.00}{##1}}}
\expandafter\def\csname PY@tok@kn\endcsname{\let\PY@bf=\textbf\def\PY@tc##1{\textcolor[rgb]{0.00,0.50,0.00}{##1}}}
\expandafter\def\csname PY@tok@kr\endcsname{\let\PY@bf=\textbf\def\PY@tc##1{\textcolor[rgb]{0.00,0.50,0.00}{##1}}}
\expandafter\def\csname PY@tok@bp\endcsname{\def\PY@tc##1{\textcolor[rgb]{0.00,0.50,0.00}{##1}}}
\expandafter\def\csname PY@tok@fm\endcsname{\def\PY@tc##1{\textcolor[rgb]{0.00,0.00,1.00}{##1}}}
\expandafter\def\csname PY@tok@vc\endcsname{\def\PY@tc##1{\textcolor[rgb]{0.10,0.09,0.49}{##1}}}
\expandafter\def\csname PY@tok@vg\endcsname{\def\PY@tc##1{\textcolor[rgb]{0.10,0.09,0.49}{##1}}}
\expandafter\def\csname PY@tok@vi\endcsname{\def\PY@tc##1{\textcolor[rgb]{0.10,0.09,0.49}{##1}}}
\expandafter\def\csname PY@tok@vm\endcsname{\def\PY@tc##1{\textcolor[rgb]{0.10,0.09,0.49}{##1}}}
\expandafter\def\csname PY@tok@sa\endcsname{\def\PY@tc##1{\textcolor[rgb]{0.73,0.13,0.13}{##1}}}
\expandafter\def\csname PY@tok@sb\endcsname{\def\PY@tc##1{\textcolor[rgb]{0.73,0.13,0.13}{##1}}}
\expandafter\def\csname PY@tok@sc\endcsname{\def\PY@tc##1{\textcolor[rgb]{0.73,0.13,0.13}{##1}}}
\expandafter\def\csname PY@tok@dl\endcsname{\def\PY@tc##1{\textcolor[rgb]{0.73,0.13,0.13}{##1}}}
\expandafter\def\csname PY@tok@s2\endcsname{\def\PY@tc##1{\textcolor[rgb]{0.73,0.13,0.13}{##1}}}
\expandafter\def\csname PY@tok@sh\endcsname{\def\PY@tc##1{\textcolor[rgb]{0.73,0.13,0.13}{##1}}}
\expandafter\def\csname PY@tok@s1\endcsname{\def\PY@tc##1{\textcolor[rgb]{0.73,0.13,0.13}{##1}}}
\expandafter\def\csname PY@tok@mb\endcsname{\def\PY@tc##1{\textcolor[rgb]{0.40,0.40,0.40}{##1}}}
\expandafter\def\csname PY@tok@mf\endcsname{\def\PY@tc##1{\textcolor[rgb]{0.40,0.40,0.40}{##1}}}
\expandafter\def\csname PY@tok@mh\endcsname{\def\PY@tc##1{\textcolor[rgb]{0.40,0.40,0.40}{##1}}}
\expandafter\def\csname PY@tok@mi\endcsname{\def\PY@tc##1{\textcolor[rgb]{0.40,0.40,0.40}{##1}}}
\expandafter\def\csname PY@tok@il\endcsname{\def\PY@tc##1{\textcolor[rgb]{0.40,0.40,0.40}{##1}}}
\expandafter\def\csname PY@tok@mo\endcsname{\def\PY@tc##1{\textcolor[rgb]{0.40,0.40,0.40}{##1}}}
\expandafter\def\csname PY@tok@ch\endcsname{\let\PY@it=\textit\def\PY@tc##1{\textcolor[rgb]{0.25,0.50,0.50}{##1}}}
\expandafter\def\csname PY@tok@cm\endcsname{\let\PY@it=\textit\def\PY@tc##1{\textcolor[rgb]{0.25,0.50,0.50}{##1}}}
\expandafter\def\csname PY@tok@cpf\endcsname{\let\PY@it=\textit\def\PY@tc##1{\textcolor[rgb]{0.25,0.50,0.50}{##1}}}
\expandafter\def\csname PY@tok@c1\endcsname{\let\PY@it=\textit\def\PY@tc##1{\textcolor[rgb]{0.25,0.50,0.50}{##1}}}
\expandafter\def\csname PY@tok@cs\endcsname{\let\PY@it=\textit\def\PY@tc##1{\textcolor[rgb]{0.25,0.50,0.50}{##1}}}

\def\PYZbs{\char`\\}
\def\PYZus{\char`\_}
\def\PYZob{\char`\{}
\def\PYZcb{\char`\}}
\def\PYZca{\char`\^}
\def\PYZam{\char`\&}
\def\PYZlt{\char`\<}
\def\PYZgt{\char`\>}
\def\PYZsh{\char`\#}
\def\PYZpc{\char`\%}
\def\PYZdl{\char`\$}
\def\PYZhy{\char`\-}
\def\PYZsq{\char`\'}
\def\PYZdq{\char`\"}
\def\PYZti{\char`\~}
% for compatibility with earlier versions
\def\PYZat{@}
\def\PYZlb{[}
\def\PYZrb{]}
\makeatother


    % Exact colors from NB
    \definecolor{incolor}{rgb}{0.0, 0.0, 0.5}
    \definecolor{outcolor}{rgb}{0.545, 0.0, 0.0}



    
    % Prevent overflowing lines due to hard-to-break entities
    \sloppy 
    % Setup hyperref package
    \hypersetup{
      breaklinks=true,  % so long urls are correctly broken across lines
      colorlinks=true,
      urlcolor=urlcolor,
      linkcolor=linkcolor,
      citecolor=citecolor,
      }
    % Slightly bigger margins than the latex defaults
    
    \geometry{verbose,tmargin=1in,bmargin=1in,lmargin=1in,rmargin=1in}
    
    

    \begin{document}
    
    
    \maketitle
    \newpage
    \tableofcontents
    \newpage
    

    
    \hypertarget{inleiding}{%
\section{Inleiding}\label{inleiding}}

    In het voorgaande onderzoek is er gekeken met meervoudige regressie, en
logistische regressie, of er verbanden zijn te vinden binnen de dataset.

Met de meervoudige regressie was het lastig om een goed model op te
stellen. De oorzaak hiervan is dat er bijna geen onderlinge correlatie
tussen de variabelen is te vinden. Het doel was om \texttt{thalanch} te
schatten a.d.h.v. de overige ratio variabelen. Het best verkregen model
hiervoor had een \(R^2\) van \(0.229\), wat in onze ogen onbruikbaar is.

Met de tweede opdracht, logistische regressie, is er gekeken of er een
verband bestaat tussen \texttt{thalach} (maximaal behaalde hartslag) and
\texttt{exang} (optreden van borstpijn tijdens de oefening). Hiervoor is
een logistisch regressiemodel opgesteld en is een correlatiecoëfficiënt
van \(0.9\) gevonden, wat aantoont dat er een sterk positief verband is
tussen de variabelen.

In de laatste opdracht, de principele componenten analyse, wordt er
gekeken welke variabelen invloed hebben om de \texttt{target} kolom
(wel/geen hartziekte). Op deze manier kan er misschien worden bepaald
welke variabelen invloed hebben of op iemand een hartziekte heeft.

    \hypertarget{data-inlezen}{%
\section{Data inlezen}\label{data-inlezen}}

    Om te beginnen met de principele componenten analyse (PCA), worden eerst
de gegevens ingeladen. Vervolgens wordt de kolomnaam voor \texttt{age}
hersteld.

    \begin{Verbatim}[commandchars=\\\{\}]
{\color{incolor}In [{\color{incolor}1}]:} df \PY{o}{=} read.csv\PY{p}{(}\PY{l+s}{\PYZsq{}}\PY{l+s}{heart.csv\PYZsq{}}\PY{p}{)}
        
        \PY{c+c1}{\PYZsh{} kolomnaam herstellen; er staat \PYZsq{}`..age\PYZsq{}}
        names \PY{o}{=} \PY{k+kp}{colnames}\PY{p}{(}df\PY{p}{)}\PY{p}{;} 
        \PY{k+kp}{names}\PY{p}{[}\PY{l+m}{1}\PY{p}{]} \PY{o}{=} \PY{l+s}{\PYZsq{}}\PY{l+s}{age\PYZsq{}} 
        \PY{k+kp}{colnames}\PY{p}{(}df\PY{p}{)} \PY{o}{=} \PY{k+kp}{names}
        
        \PY{k+kp}{head}\PY{p}{(}df\PY{p}{)}
        \PY{k+kp}{print}\PY{p}{(}\PY{k+kp}{paste}\PY{p}{(}\PY{l+s}{\PYZsq{}}\PY{l+s}{Er zijn\PYZsq{}}\PY{p}{,} \PY{k+kp}{nrow}\PY{p}{(}df\PY{p}{)}\PY{p}{,} \PY{l+s}{\PYZsq{}}\PY{l+s}{rijen, en\PYZsq{}}\PY{p}{,} \PY{k+kp}{length}\PY{p}{(}df\PY{p}{)}\PY{p}{,} \PY{l+s}{\PYZsq{}}\PY{l+s}{ kolommen.\PYZsq{}}\PY{p}{)}\PY{p}{)}
\end{Verbatim}

    \begin{tabular}{r|llllllllllllll}
 age & sex & cp & trestbps & chol & fbs & restecg & thalach & exang & oldpeak & slope & ca & thal & target\\
\hline
	 63  & 1   & 3   & 145 & 233 & 1   & 0   & 150 & 0   & 2.3 & 0   & 0   & 1   & 1  \\
	 37  & 1   & 2   & 130 & 250 & 0   & 1   & 187 & 0   & 3.5 & 0   & 0   & 2   & 1  \\
	 41  & 0   & 1   & 130 & 204 & 0   & 0   & 172 & 0   & 1.4 & 2   & 0   & 2   & 1  \\
	 56  & 1   & 1   & 120 & 236 & 0   & 1   & 178 & 0   & 0.8 & 2   & 0   & 2   & 1  \\
	 57  & 0   & 0   & 120 & 354 & 0   & 1   & 163 & 1   & 0.6 & 2   & 0   & 2   & 1  \\
	 57  & 1   & 0   & 140 & 192 & 0   & 1   & 148 & 0   & 0.4 & 1   & 0   & 1   & 1  \\
\end{tabular}


    
    \begin{Verbatim}[commandchars=\\\{\}]
[1] "Er zijn 303 rijen, en 14  kolommen."

    \end{Verbatim}

    Er is te zien dat er 303 rijen (aantal samples, \(N=303\)) zijn en 14
kolommen (aantal variabelen, \(v=14\)) in de dataset zitten.

Omdat er voorspeld moet worden of iemand een hartziekte heeft, wordt de
kolom \texttt{target} niet in het model opgenomen. Echter wordt
\texttt{target} wel gebruikt als groepering voor de samples. Alle
binaire variabelen worden wel opgenomen in het model.

Tot slot worden er alvast een aantal functies gedeclareerd, die later
gebruikt worden.

    \begin{Verbatim}[commandchars=\\\{\}]
{\color{incolor}In [{\color{incolor}2}]:} \PY{c+c1}{\PYZsh{} Calculate the trace of a matrix M.}
        trace \PY{o}{=} \PY{k+kr}{function}\PY{p}{(}M\PY{p}{)} \PY{p}{\PYZob{}} \PY{p}{(}\PY{k+kp}{sum}\PY{p}{(}\PY{k+kp}{diag}\PY{p}{(}M\PY{p}{)}\PY{p}{)}\PY{p}{)} \PY{p}{\PYZcb{}}
        
        \PY{c+c1}{\PYZsh{} Calculate the variance of a matrix M.}
        matrix.var \PY{o}{=} \PY{k+kr}{function}\PY{p}{(}M\PY{p}{)} \PY{p}{\PYZob{}} \PY{p}{(}\PY{k+kp}{trace}\PY{p}{(}\PY{k+kp}{t}\PY{p}{(}M\PY{p}{)} \PY{o}{\PYZpc{}*\PYZpc{}} M\PY{p}{)} \PY{o}{/} \PY{p}{(}\PY{k+kp}{ncol}\PY{p}{(}M\PY{p}{)} \PY{o}{*} \PY{k+kp}{nrow}\PY{p}{(}M\PY{p}{)}\PY{p}{)}\PY{p}{)} \PY{p}{\PYZcb{}}
        
        \PY{c+c1}{\PYZsh{} Calculate the standard deviation of a matrix M.}
        matrix.sd  \PY{o}{=} \PY{k+kr}{function}\PY{p}{(}M\PY{p}{)} \PY{p}{\PYZob{}} \PY{p}{(}\PY{k+kp}{sqrt}\PY{p}{(}matrix.var\PY{p}{(}M\PY{p}{)}\PY{p}{)}\PY{p}{)} \PY{p}{\PYZcb{}}
        
        \PY{c+c1}{\PYZsh{} Calculate the mean for every row of a matrix M.}
        rowMeans \PY{o}{=} \PY{k+kr}{function}\PY{p}{(}M\PY{p}{)} \PY{p}{\PYZob{}} \PY{p}{(}\PY{k+kp}{rowSums}\PY{p}{(}M\PY{p}{)} \PY{o}{/} \PY{k+kp}{ncol}\PY{p}{(}M\PY{p}{)}\PY{p}{)} \PY{p}{\PYZcb{}}
        
        \PY{c+c1}{\PYZsh{} Calculate the mean for every column of a matrix M.}
        colMeans \PY{o}{=} \PY{k+kr}{function}\PY{p}{(}M\PY{p}{)} \PY{p}{\PYZob{}} \PY{p}{(}\PY{k+kp}{colSums}\PY{p}{(}M\PY{p}{)} \PY{o}{/} \PY{k+kp}{nrow}\PY{p}{(}M\PY{p}{)}\PY{p}{)} \PY{p}{\PYZcb{}}
\end{Verbatim}

    \hypertarget{beschrijving-van-de-kolommen}{%
\section{Beschrijving van de
kolommen}\label{beschrijving-van-de-kolommen}}

Op de website waar de dataset is verkregen, is de volgende informatie
gevonden over de kolommen. De dataset bevat in totaal 14 kolommen.

\begin{enumerate}
\def\labelenumi{\arabic{enumi}.}
\tightlist
\item
  \texttt{age}: leeftijd. (Ratio)
\item
  \texttt{sex}: geslacht. (Nominaal)
\item
  \texttt{cp}: chest pain type (4 values). (Nominaal)
\item
  \texttt{trestbps}: resting blood pressure. (Ratio)
\item
  \texttt{chol}: serum cholestoral in mg/dl. (Ratio)
\item
  \texttt{fbs}: fasting blood sugar \textgreater{} 120 mg/dl. (Nominaal)
\item
  \texttt{restecg}: resting electrocardiographic results (values 0, 1,
  2). (Nominaal)
\item
  \texttt{thalanch}: maximum heartrate achieved. (Ratio)
\item
  \texttt{exang}: exercise induced angina. (Nominaal)
\item
  \texttt{oldpeak}: ST depression induced by exercise relative to test.
  (Ratio)
\item
  \texttt{slope}: the slope of the peak exercise ST segment. (Nominaal)
\item
  \texttt{ca}: number of major vessels (0-3) color by flourosopy.
  (Ordinaal)
\item
  \texttt{thal}: 3 = normal, 6 = fixed defect, 7 = reversable defect.
  (Nominaal)
\item
  \texttt{target}: indicated if someone has heart disease, 0 = false, 1
  = true. (Nominaal)
\end{enumerate}

Van de meeste variabelen is het niet echt duidelijk waar het voor staat.
Neem bijvoorbeeld \texttt{oldpeak}, in dit geval is
\texttt{ST\ depression} een fenomeen dat voorkomt in een ECG. Een ECG is
een electrocardiogram, een grafiek van de electrische activiteit van het
hart. In dit geval wordt met \texttt{ST\ depression} een bepaalt patroon
in de grafiek bedoeld. {[}1{]} Een aantal van deze patronen kunnen
duiden op een hartziekte. {[}2{]}

Een andere variabele is \texttt{exang}, wat aangeeft of er angina is
voorkomen tijdens de oefening. Een angina is een pijn of
oncomfortabelheid in de borst, mogelijk veroorzaakt doordat er te weinig
zuurstof-rijk bloed bij die spier komt. Angina is echter geen ziekte,
maar mogelijk een symptoom van een onderliggend hartprobleem. {[}3{]}

    \hypertarget{verdelingen-van-de-gegevens}{%
\section{Verdelingen van de
gegevens}\label{verdelingen-van-de-gegevens}}

    In deze paragraaf wordt er gekeken naar de verdelingen van de gegeven
binnen de verschillende variabelen. Indien de verdelingen sterk qua
schaal verschillen, is het wenselijk om de gegevens te standaardiseren.
Op deze manier wordt er voorkomen, dat de variabele met een grote
schaal, zwaarder worden meegeteld binnen het model.

\hypertarget{nominaleordinale-variabelen}{%
\subsection{Nominale/ordinale
variabelen}\label{nominaleordinale-variabelen}}

    In het onderstaande figuur staan staafdiagrammen voor elk van de
nominale/ordinale variabelen.

    \begin{Verbatim}[commandchars=\\\{\}]
{\color{incolor}In [{\color{incolor}3}]:} par\PY{p}{(}mfrow\PY{o}{=}\PY{k+kt}{c}\PY{p}{(}\PY{l+m}{3}\PY{p}{,}\PY{l+m}{3}\PY{p}{)}\PY{p}{)}
        barplot\PY{p}{(}\PY{k+kp}{table}\PY{p}{(}df\PY{o}{\PYZdl{}}sex\PY{p}{)}\PY{p}{,} main\PY{o}{=}\PY{l+s}{\PYZdq{}}\PY{l+s}{df\PYZdl{}sex\PYZdq{}}\PY{p}{)}
        barplot\PY{p}{(}\PY{k+kp}{table}\PY{p}{(}df\PY{o}{\PYZdl{}}cp\PY{p}{)}\PY{p}{,} main\PY{o}{=}\PY{l+s}{\PYZdq{}}\PY{l+s}{df\PYZdl{}cp\PYZdq{}}\PY{p}{)}
        barplot\PY{p}{(}\PY{k+kp}{table}\PY{p}{(}df\PY{o}{\PYZdl{}}fbs\PY{p}{)}\PY{p}{,} main\PY{o}{=}\PY{l+s}{\PYZdq{}}\PY{l+s}{df\PYZdl{}fbs\PYZdq{}}\PY{p}{)}
        barplot\PY{p}{(}\PY{k+kp}{table}\PY{p}{(}df\PY{o}{\PYZdl{}}restecg\PY{p}{)}\PY{p}{,} main\PY{o}{=}\PY{l+s}{\PYZdq{}}\PY{l+s}{df\PYZdl{}restecg\PYZdq{}}\PY{p}{)}
        barplot\PY{p}{(}\PY{k+kp}{table}\PY{p}{(}df\PY{o}{\PYZdl{}}exang\PY{p}{)}\PY{p}{,} main\PY{o}{=}\PY{l+s}{\PYZdq{}}\PY{l+s}{df\PYZdl{}exang\PYZdq{}}\PY{p}{)}
        barplot\PY{p}{(}\PY{k+kp}{table}\PY{p}{(}df\PY{o}{\PYZdl{}}slope\PY{p}{)}\PY{p}{,} main\PY{o}{=}\PY{l+s}{\PYZdq{}}\PY{l+s}{df\PYZdl{}slope\PYZdq{}}\PY{p}{)}
        barplot\PY{p}{(}\PY{k+kp}{table}\PY{p}{(}df\PY{o}{\PYZdl{}}ca\PY{p}{)}\PY{p}{,} main\PY{o}{=}\PY{l+s}{\PYZdq{}}\PY{l+s}{df\PYZdl{}ca\PYZdq{}}\PY{p}{)}
        barplot\PY{p}{(}\PY{k+kp}{table}\PY{p}{(}df\PY{o}{\PYZdl{}}thal\PY{p}{)}\PY{p}{,} main\PY{o}{=}\PY{l+s}{\PYZdq{}}\PY{l+s}{df\PYZdl{}thal\PYZdq{}}\PY{p}{)}
        barplot\PY{p}{(}\PY{k+kp}{table}\PY{p}{(}df\PY{o}{\PYZdl{}}target\PY{p}{)}\PY{p}{,} main\PY{o}{=}\PY{l+s}{\PYZdq{}}\PY{l+s}{df\PYZdl{}targer\PYZdq{}}\PY{p}{)}
\end{Verbatim}

    \begin{center}
    \adjustimage{max size={0.9\linewidth}{0.9\paperheight}}{output_11_0.png}
    \end{center}
    { \hspace*{\fill} \\}
    
    Hier valt het op dat er een combinatie is van binaire variabelen, en
variabelen met meerdere schaalpunten. Als er niet gestandaardiseerd
wordt, dan worden de variabelen met meerdere groepen zwaarder meegeteld
in het model.

    \hypertarget{ratio-variabelen}{%
\subsection{Ratio variabelen}\label{ratio-variabelen}}

    In het onderstaande figuur staan histogrammen van de interval/ratio
variabelen.

    \begin{Verbatim}[commandchars=\\\{\}]
{\color{incolor}In [{\color{incolor}4}]:} par\PY{p}{(}mfrow\PY{o}{=}\PY{k+kt}{c}\PY{p}{(}\PY{l+m}{3}\PY{p}{,} \PY{l+m}{2}\PY{p}{)}\PY{p}{)}
        hist\PY{p}{(}df\PY{o}{\PYZdl{}}age\PY{p}{)}
        hist\PY{p}{(}df\PY{o}{\PYZdl{}}trestbps\PY{p}{)}
        hist\PY{p}{(}df\PY{o}{\PYZdl{}}\PY{k+kp}{chol}\PY{p}{)}
        hist\PY{p}{(}df\PY{o}{\PYZdl{}}thalach\PY{p}{)}
        hist\PY{p}{(}df\PY{o}{\PYZdl{}}oldpeak\PY{p}{)}
\end{Verbatim}

    \begin{center}
    \adjustimage{max size={0.9\linewidth}{0.9\paperheight}}{output_15_0.png}
    \end{center}
    { \hspace*{\fill} \\}
    
    Het valt op dat de schaal waarop de waarnemingen zijn gemeten sterk
verschillen. Vanuit dit oogpunt is het verstandig om de gegevens te
standaardiseren.

    \hypertarget{centreren-en-standaardiseren}{%
\section{Centreren en
standaardiseren}\label{centreren-en-standaardiseren}}

    Vanuit de vorige paragraaf is gebleken dat het verstandig is om de
gegevens, naast het centreren, ook te standaardiseren. Het centreren en
standaardiseren, wordt ook wel normalizeren genoemd. Hiervoor wordt de
\(z\)-score berekend. De formule hiervoor is
\(z = \frac{x-\bar{x}}{s}\), waarbij \(\bar{x}\) het gemiddelde is en
\(s\) de standaardafwijking.

Eerst wordt de matrix \(X\) opgesteld met de gegevens die relevant zijn.
Daarna worden de gegevens gecentreerd, aangeduidt met \(X_c\). Door de
gegevens te centreren, worden alle hieropvolgende berekeningen
vereenvoudigd. Daarnaast worden de gegevens gestandaardiseerd,
aangeduidt met \(X_{cs}\), zodat elke variabele evenzwaar meetelt binnen
het model.

    \begin{Verbatim}[commandchars=\\\{\}]
{\color{incolor}In [{\color{incolor}5}]:} X \PY{o}{=} \PY{k+kp}{data.matrix}\PY{p}{(}df\PY{p}{[}\PY{p}{,}\PY{l+m}{1}\PY{o}{:}\PY{l+m}{13}\PY{p}{]}\PY{p}{)} \PY{c+c1}{\PYZsh{} alle gegevens behalve de \PYZsq{}target\PYZsq{} kolom}
        m \PY{o}{=} \PY{k+kp}{apply}\PY{p}{(}X\PY{p}{,}\PY{l+m}{2}\PY{p}{,}\PY{k+kp}{mean}\PY{p}{)} \PY{c+c1}{\PYZsh{} gemiddelde berekenen, voor elke kolom}
        s \PY{o}{=} \PY{k+kp}{apply}\PY{p}{(}X\PY{p}{,}\PY{l+m}{2}\PY{p}{,}sd\PY{p}{)} \PY{c+c1}{\PYZsh{} standaard deviatie berekenen, voor elke kolom}
        Xc \PY{o}{=} \PY{k+kp}{sweep}\PY{p}{(}X\PY{p}{,}\PY{l+m}{2}\PY{p}{,}m\PY{p}{,}\PY{l+s}{\PYZdq{}}\PY{l+s}{\PYZhy{}\PYZdq{}}\PY{p}{)} \PY{c+c1}{\PYZsh{} centreren}
        Xcs \PY{o}{=} \PY{k+kp}{sweep}\PY{p}{(}Xc\PY{p}{,}\PY{l+m}{2}\PY{p}{,}s\PY{p}{,}\PY{l+s}{\PYZdq{}}\PY{l+s}{/\PYZdq{}}\PY{p}{)} \PY{c+c1}{\PYZsh{} standaardiseren}
\end{Verbatim}

    In de onderstaande tabel staat een klein voorbeeld van hoe de gegevens
van \(X_{cs}\) er nu uitzien. Voor elke kolom is nu het gemiddelde
\(0\).

    \begin{Verbatim}[commandchars=\\\{\}]
{\color{incolor}In [{\color{incolor}6}]:} \PY{k+kp}{head}\PY{p}{(}Xcs\PY{p}{[}\PY{p}{,}\PY{l+m}{1}\PY{o}{:}\PY{l+m}{5}\PY{p}{]}\PY{p}{)}
\end{Verbatim}

    \begin{tabular}{lllll}
 age & sex & cp & trestbps & chol\\
\hline
	  0.9506240  &  0.6798805  &  1.96986425 &  0.76269408 & -0.25591036\\
	 -1.9121497  &  0.6798805  &  1.00092128 & -0.09258463 &  0.07208025\\
	 -1.4717230  & -1.4659924  &  0.03197832 & -0.09258463 & -0.81542377\\
	  0.1798773  &  0.6798805  &  0.03197832 & -0.66277043 & -0.19802967\\
	  0.2899839  & -1.4659924  & -0.93696465 & -0.66277043 &  2.07861109\\
	  0.2899839  &  0.6798805  & -0.93696465 &  0.47760118 & -1.04694656\\
\end{tabular}


    
    \hypertarget{berekenen-van-principele-componenten}{%
\section{Berekenen van principele
componenten}\label{berekenen-van-principele-componenten}}

    In deze paragraaf worden de principele componenten bepaald voor de
gestandaardiseerde gegevens. Het eerste principele component \(PC_1\),
is de eigenvector \(p_1'=[\beta_1,\beta_2,\ldots,\beta_{13}]\) met de
hoogste eigenwaarde van de covariantie matrix van \(X_{cs}\). De
covariantiematrix wordt als volgt bepaald:

\[ \textrm{cov}(\mathbf{X}) = \dfrac{1}{N-1} \mathbf{X}^T\mathbf{X}, \]

maar gelukkig kan dit ook in R met \texttt{cov(X)}. Vervolgens wordt van
de covariantiematrix van \(X_{cs}\), de eigenvectoren en eigenwaarden
bepaald. Om de eigenvectoren en eigenwaarden te vinden, is er de functie
\texttt{eigen(X)}. De vectoren worden opgeslagen in
\texttt{eigen(X)\$vectors} en staan al in aflopende volgorde.

    \begin{Verbatim}[commandchars=\\\{\}]
{\color{incolor}In [{\color{incolor}7}]:} p \PY{o}{=} \PY{k+kp}{eigen}\PY{p}{(}cor\PY{p}{(}Xcs\PY{p}{)}\PY{p}{)}\PY{o}{\PYZdl{}}vectors \PY{c+c1}{\PYZsh{} eigenvectoren van de covariantiematrix van Xcs }
                                    \PY{c+c1}{\PYZsh{} bepalen, dit zijn ook direct de PCAs}
        
        \PY{k+kp}{colnames}\PY{p}{(}p\PY{p}{)} \PY{o}{=} \PY{k+kt}{c}\PY{p}{(}\PY{l+s}{\PYZsq{}}\PY{l+s}{PC1\PYZsq{}}\PY{p}{,} \PY{l+s}{\PYZsq{}}\PY{l+s}{PC2\PYZsq{}}\PY{p}{,} \PY{l+s}{\PYZsq{}}\PY{l+s}{PC3\PYZsq{}}\PY{p}{,} \PY{l+s}{\PYZsq{}}\PY{l+s}{PC4\PYZsq{}}\PY{p}{,} \PY{l+s}{\PYZsq{}}\PY{l+s}{PC5\PYZsq{}}\PY{p}{,} \PY{l+s}{\PYZsq{}}\PY{l+s}{PC6\PYZsq{}}\PY{p}{,} \PY{l+s}{\PYZsq{}}\PY{l+s}{PC7\PYZsq{}}\PY{p}{,} 
                        \PY{l+s}{\PYZsq{}}\PY{l+s}{PC8\PYZsq{}}\PY{p}{,} \PY{l+s}{\PYZsq{}}\PY{l+s}{PC9\PYZsq{}}\PY{p}{,} \PY{l+s}{\PYZsq{}}\PY{l+s}{PC10\PYZsq{}}\PY{p}{,} \PY{l+s}{\PYZsq{}}\PY{l+s}{PC11\PYZsq{}}\PY{p}{,} \PY{l+s}{\PYZsq{}}\PY{l+s}{PC12\PYZsq{}}\PY{p}{,} \PY{l+s}{\PYZsq{}}\PY{l+s}{PC13\PYZsq{}}\PY{p}{)}
        
        \PY{k+kp}{rownames}\PY{p}{(}p\PY{p}{)} \PY{o}{=} \PY{k+kp}{colnames}\PY{p}{(}df\PY{p}{[}\PY{p}{,}\PY{l+m}{1}\PY{o}{:}\PY{l+m}{13}\PY{p}{]}\PY{p}{)}
        
        p\PY{p}{[}\PY{p}{,}\PY{l+m}{1}\PY{o}{:}\PY{l+m}{6}\PY{p}{]} \PY{c+c1}{\PYZsh{} eerste x PCAs weergeven}
        \PY{k+kp}{paste}\PY{p}{(}\PY{l+s}{\PYZsq{}}\PY{l+s}{Totaal:\PYZsq{}}\PY{p}{,} \PY{k+kp}{ncol}\PY{p}{(}p\PY{p}{)}\PY{p}{,} \PY{l+s}{\PYZsq{}}\PY{l+s}{kolommen\PYZsq{}}\PY{p}{)}
\end{Verbatim}

    \begin{tabular}{r|llllll}
  & PC1 & PC2 & PC3 & PC4 & PC5 & PC6\\
\hline
	age &  0.31420252 & -0.40614872 & -0.09407661 &  0.02066180 &  0.30715312 & -0.12829615\\
	sex &  0.09083783 &  0.37779171 &  0.55484915 &  0.25530873 & -0.05070440 &  0.05496875\\
	cp & -0.27460749 & -0.29726609 &  0.35697431 & -0.28790041 & -0.16317945 & -0.19341117\\
	trestbps &  0.18392019 & -0.43818675 &  0.20384930 & -0.02260103 & -0.18813809 & -0.17945982\\
	chol &  0.11737503 & -0.36451402 & -0.40782498 &  0.34340982 & -0.32006670 & -0.10472957\\
	fbs &  0.07363999 & -0.31743328 &  0.48173624 &  0.06860532 &  0.23344184 &  0.24961364\\
	restecg & -0.12772792 &  0.22088181 & -0.08919083 & -0.26609555 &  0.39366727 & -0.66681339\\
	thalach & -0.41649811 & -0.07787618 &  0.15825529 &  0.18412539 & -0.32328431 & -0.12098445\\
	exang &  0.36126745 &  0.26311790 & -0.12635610 &  0.11505621 & -0.03453568 &  0.23069914\\
	oldpeak &  0.41963899 &  0.05225497 &  0.11034290 & -0.32629597 & -0.25057927 & -0.17007984\\
	slope & -0.37977222 & -0.04837415 & -0.07381839 &  0.49484894 &  0.24682275 & -0.06406935\\
	ca &  0.27326172 & -0.09414721 &  0.18356934 &  0.32801632 &  0.43536515 & -0.18210750\\
	thal &  0.22202375 &  0.20072042 &  0.12501113 &  0.38919138 & -0.33195049 & -0.50885654\\
\end{tabular}


    
    'Totaal: 13 kolommen'

    
    Er is te zien dat er in totaal \(13\) principele componenten zijn,
namelijk het aantal variabelen binnen de gegevens. Vanuit de tabel kan
de vector voor \(PC_1\) worden afgelezen, namelijk:

\[ p_1 = \begin{bmatrix} 0.314 \\ 0.090 \\ -0.274 \\ 0.183 \\ 0.117 \\ 0.073 \\ -0.127 \\ -0.416 \\ 0.361 \\ 0.419 \\ -0.379 \\ 0.273 \\ 0.222 \end{bmatrix}. \]

    Alle elementen in \(p_1\) worden ook wel \emph{loadings} genoemd. Aan de
loadings valt te zien dat de waarde \(0.419\) de hoogste is. Dit
betekend dat \texttt{oldpeak} het meeste invloed heeft op \(PC_1\).
Daarnaast heeft \texttt{thalach} de hoogste negatieve invloed op
\(PC_1\), met een waarde van \(-0.416\). Om te controleren of dit de
juiste vector voor \(PC_1\) is, kan er met R worden gekeken met
\texttt{prcomp(...)}:

    \begin{Verbatim}[commandchars=\\\{\}]
{\color{incolor}In [{\color{incolor}8}]:} \PY{k+kp}{data.matrix}\PY{p}{(}prcomp\PY{p}{(}df\PY{p}{[}\PY{p}{,}\PY{l+m}{1}\PY{o}{:}\PY{l+m}{13}\PY{p}{]}\PY{p}{,} scale\PY{o}{=}\PY{n+nb+bp}{T}\PY{p}{)}\PY{o}{\PYZdl{}}rotation\PY{p}{[}\PY{p}{,}\PY{l+m}{1}\PY{p}{]}\PY{p}{)} \PY{c+c1}{\PYZsh{} PCA1}
        \PY{c+c1}{\PYZsh{} data.matrix wordt gebruikt voor de opmaak hieronder.}
\end{Verbatim}

    \begin{tabular}{r|l}
	age & -0.31420252\\
	sex & -0.09083783\\
	cp &  0.27460749\\
	trestbps & -0.18392019\\
	chol & -0.11737503\\
	fbs & -0.07363999\\
	restecg &  0.12772792\\
	thalach &  0.41649811\\
	exang & -0.36126745\\
	oldpeak & -0.41963899\\
	slope &  0.37977222\\
	ca & -0.27326172\\
	thal & -0.22202375\\
\end{tabular}


    
    Dit komt overeen met de eigenvector met de grootste eigenwaarde van de
covariantiematrix van \(X_{cs}\), alleen het teken is omgedraaid. In de
volgende paragraaf wordt hiervoor een oplossing besproken.

    \hypertarget{rotatieambivalentie}{%
\section{Rotatieambivalentie}\label{rotatieambivalentie}}

Omdat eigenvectoren allemaal loodrecht op elkaar staan, zijn er altijd
twee vectoren mogelijk. Om dezelfde resultaten te krijgen als
\texttt{prcomp(...)}, is het nodig om alle eigenvectoren om te keren.
Dit concept wordt \emph{rotatieambivalentie} genoemdt. Dit is eenvoudig
op te lossen met \(p := -p\).

    \begin{Verbatim}[commandchars=\\\{\}]
{\color{incolor}In [{\color{incolor}9}]:} p \PY{o}{=} \PY{o}{\PYZhy{}}p
\end{Verbatim}

    \hypertarget{biplot-score-plot-loading-plot}{%
\section{Biplot: score plot + loading
plot}\label{biplot-score-plot-loading-plot}}

    Om een bi-plot te maken, zijn er twee grafieken nodig. Er is een score
plot tussen \(PC_1\) en \(PC_2\), en een loading plot.

Om de scores \(t_1\) voor \(PC_1\) te bepalen, worden alle waarnemingen
loodrecht op \(p_1\) geprojecteerd. De afstand vanaf de oorsprong \(O\)
tot aan de loodrechte projectie op \(p_1\) is het nieuwe x-coordinaat
van de waarneming in het \(p_1, p_2\) vlak. Ditzelfde wordt gedaan om de
scores \(t_2\) te bepalen voor \(p_2\), en dient als y-coordinaat.

    \begin{Verbatim}[commandchars=\\\{\}]
{\color{incolor}In [{\color{incolor}10}]:} t1 \PY{o}{=} Xcs \PY{o}{\PYZpc{}*\PYZpc{}} p\PY{p}{[}\PY{p}{,}\PY{l+m}{1}\PY{p}{]} \PY{c+c1}{\PYZsh{} projectie op p1}
         t2 \PY{o}{=} Xcs \PY{o}{\PYZpc{}*\PYZpc{}} p\PY{p}{[}\PY{p}{,}\PY{l+m}{2}\PY{p}{]} \PY{c+c1}{\PYZsh{} projectie op p2}
         t \PY{o}{=} Xcs \PY{o}{\PYZpc{}*\PYZpc{}} p \PY{c+c1}{\PYZsh{} voor alle p}
\end{Verbatim}

    Als deze nieuwe scores in een grafiek worden weergegeven, heet dit een
score plot. De tweede grafiek, de loading plot, laat zien hoe de
loadings van \(PC_1\) en \(PC_2\) bepalen waar de waarnemingen in de
score plot belanden. De loadings mogen geschaald worden voor een
weergave in een bi-plot. Het gaat is immers de onderline verhouding die
belangrijk is. Met de onderstaande code, worden beide grafieken geplot
in een bi-plot:

    \begin{Verbatim}[commandchars=\\\{\}]
{\color{incolor}In [{\color{incolor}11}]:} \PY{c+c1}{\PYZsh{} score plot}
         plot\PY{p}{(}t1\PY{p}{,} t2\PY{p}{,} pch\PY{o}{=}\PY{k+kc}{NA}\PY{p}{,} xlab\PY{o}{=}\PY{l+s}{\PYZdq{}}\PY{l+s}{PCA1\PYZdq{}}\PY{p}{,} ylab\PY{o}{=}\PY{l+s}{\PYZdq{}}\PY{l+s}{PCA2\PYZdq{}} \PY{p}{)} 
         abline\PY{p}{(}h\PY{o}{=}\PY{l+m}{0}\PY{p}{,}lty\PY{o}{=}\PY{l+m}{1}\PY{p}{)} 
         abline\PY{p}{(}v\PY{o}{=}\PY{l+m}{0}\PY{p}{,}lty\PY{o}{=}\PY{l+m}{1}\PY{p}{)}
         title\PY{p}{(}main\PY{o}{=}\PY{l+s}{\PYZsq{}}\PY{l+s}{PCA bi\PYZhy{}plot\PYZsq{}}\PY{p}{)}
         grid\PY{p}{(}\PY{p}{)}
         
         \PY{c+c1}{\PYZsh{} zet de groepnamen in de plot + rood markeren als target=1}
         text\PY{p}{(}t1\PY{p}{,} t2\PY{p}{,} labels\PY{o}{=}df\PY{o}{\PYZdl{}}target\PY{p}{,} pos\PY{o}{=}\PY{l+m}{1}\PY{p}{,} xpd\PY{o}{=}\PY{k+kc}{NA}\PY{p}{,} col\PY{o}{=}\PY{k+kt}{c}\PY{p}{(}\PY{l+s}{\PYZsq{}}\PY{l+s}{black\PYZsq{}}\PY{p}{,} \PY{l+s}{\PYZsq{}}\PY{l+s}{red\PYZsq{}}\PY{p}{)}\PY{p}{[}df\PY{o}{\PYZdl{}}target\PY{l+m}{+1}\PY{p}{]}\PY{p}{)}
         
         \PY{c+c1}{\PYZsh{} loading plot}
         scale \PY{o}{=} \PY{l+m}{7}
         arrows\PY{p}{(}\PY{l+m}{0}\PY{p}{,}\PY{l+m}{0}\PY{p}{,}p\PY{p}{[}\PY{p}{,}\PY{l+m}{1}\PY{p}{]} \PY{o}{*} \PY{k+kp}{scale}\PY{p}{,} p\PY{p}{[}\PY{p}{,}\PY{l+m}{2}\PY{p}{]} \PY{o}{*} \PY{k+kp}{scale}\PY{p}{,} col\PY{o}{=}\PY{l+s}{\PYZsq{}}\PY{l+s}{blue\PYZsq{}}\PY{p}{,} lwd\PY{o}{=}\PY{l+m}{1}\PY{p}{)}
         text\PY{p}{(}p\PY{p}{[}\PY{p}{,}\PY{l+m}{1}\PY{p}{]} \PY{o}{*} \PY{k+kp}{scale}\PY{p}{,} p\PY{p}{[}\PY{p}{,}\PY{l+m}{2}\PY{p}{]} \PY{o}{*} \PY{k+kp}{scale}\PY{p}{,} labels\PY{o}{=}\PY{k+kp}{colnames}\PY{p}{(}Xcs\PY{p}{)}\PY{p}{,} col\PY{o}{=}\PY{l+s}{\PYZsq{}}\PY{l+s}{blue\PYZsq{}}\PY{p}{,} pos\PY{o}{=}\PY{l+m}{4}\PY{p}{)}
\end{Verbatim}

    \begin{center}
    \adjustimage{max size={0.9\linewidth}{0.9\paperheight}}{output_35_0.png}
    \end{center}
    { \hspace*{\fill} \\}
    
    Wat direct opvalt is dat bijna alle waarnemingen die een hartziekte
hebben, een positieve waarde hebben voor \(PC_1\). Eerder bij de
loadings werdt al duidelijk dat \texttt{oldpeak} en \texttt{thalach} het
meeste invloed uit oefenenen, en ook hier valt te zien richting hiervan
bijna horizontaal is. Twee andere variabelen die ook een grote loading
hebben zijn \texttt{slope} en \texttt{exang}, maar dit zijn nominale
variabelen.

Ditzelfde kan ook gedaan worden voor \(PC_2\) en \(PC_3\):

    \begin{Verbatim}[commandchars=\\\{\}]
{\color{incolor}In [{\color{incolor}12}]:} \PY{c+c1}{\PYZsh{} score plot}
         plot\PY{p}{(}\PY{k+kp}{t}\PY{p}{[}\PY{p}{,}\PY{l+m}{2}\PY{p}{]}\PY{p}{,} \PY{k+kp}{t}\PY{p}{[}\PY{p}{,}\PY{l+m}{3}\PY{p}{]}\PY{p}{,} pch\PY{o}{=}\PY{k+kc}{NA}\PY{p}{,} xlab\PY{o}{=}\PY{l+s}{\PYZdq{}}\PY{l+s}{PCA2\PYZdq{}}\PY{p}{,} ylab\PY{o}{=}\PY{l+s}{\PYZdq{}}\PY{l+s}{PCA3\PYZdq{}} \PY{p}{)} 
         abline\PY{p}{(}h\PY{o}{=}\PY{l+m}{0}\PY{p}{,}lty\PY{o}{=}\PY{l+m}{1}\PY{p}{)} 
         abline\PY{p}{(}v\PY{o}{=}\PY{l+m}{0}\PY{p}{,}lty\PY{o}{=}\PY{l+m}{1}\PY{p}{)}
         title\PY{p}{(}main\PY{o}{=}\PY{l+s}{\PYZsq{}}\PY{l+s}{PCA bi\PYZhy{}plot\PYZsq{}}\PY{p}{)}
         grid\PY{p}{(}\PY{p}{)}
         
         \PY{c+c1}{\PYZsh{} zet de groepnamen in de plot + rood markeren als target=1}
         text\PY{p}{(}\PY{k+kp}{t}\PY{p}{[}\PY{p}{,}\PY{l+m}{2}\PY{p}{]}\PY{p}{,} \PY{k+kp}{t}\PY{p}{[}\PY{p}{,}\PY{l+m}{3}\PY{p}{]}\PY{p}{,} labels\PY{o}{=}df\PY{o}{\PYZdl{}}target\PY{p}{,} pos\PY{o}{=}\PY{l+m}{1}\PY{p}{,} 
              xpd\PY{o}{=}\PY{k+kc}{NA}\PY{p}{,} col\PY{o}{=}\PY{k+kt}{c}\PY{p}{(}\PY{l+s}{\PYZsq{}}\PY{l+s}{black\PYZsq{}}\PY{p}{,} \PY{l+s}{\PYZsq{}}\PY{l+s}{red\PYZsq{}}\PY{p}{)}\PY{p}{[}df\PY{o}{\PYZdl{}}target\PY{l+m}{+1}\PY{p}{]}\PY{p}{)}
         
         \PY{c+c1}{\PYZsh{} loading plot}
         scale \PY{o}{=} \PY{l+m}{7}
         arrows\PY{p}{(}\PY{l+m}{0}\PY{p}{,}\PY{l+m}{0}\PY{p}{,}p\PY{p}{[}\PY{p}{,}\PY{l+m}{2}\PY{p}{]} \PY{o}{*} \PY{k+kp}{scale}\PY{p}{,} p\PY{p}{[}\PY{p}{,}\PY{l+m}{3}\PY{p}{]} \PY{o}{*} \PY{k+kp}{scale}\PY{p}{,} col\PY{o}{=}\PY{l+s}{\PYZsq{}}\PY{l+s}{blue\PYZsq{}}\PY{p}{,} lwd\PY{o}{=}\PY{l+m}{1}\PY{p}{)}
         text\PY{p}{(}p\PY{p}{[}\PY{p}{,}\PY{l+m}{2}\PY{p}{]} \PY{o}{*} \PY{k+kp}{scale}\PY{p}{,} p\PY{p}{[}\PY{p}{,}\PY{l+m}{3}\PY{p}{]} \PY{o}{*} \PY{k+kp}{scale}\PY{p}{,} labels\PY{o}{=}\PY{k+kp}{colnames}\PY{p}{(}Xcs\PY{p}{)}\PY{p}{,} 
              col\PY{o}{=}\PY{l+s}{\PYZsq{}}\PY{l+s}{blue\PYZsq{}}\PY{p}{,} pos\PY{o}{=}\PY{l+m}{4}\PY{p}{)}
\end{Verbatim}

    \begin{center}
    \adjustimage{max size={0.9\linewidth}{0.9\paperheight}}{output_37_0.png}
    \end{center}
    { \hspace*{\fill} \\}
    
    Hier valt te zien dat er niet echt duidelijk kan worden gemaakt wanneer
iemand een hartziekte heeft.

    \hypertarget{controle-met-r}{%
    	\newpage
\section{Biplot: score plot + loading plot (controle met R)}\label{controle-met-r}}

    Om te controleren of alles tot zover goed is gegaan, kan er met R snel
een bi-plot worden gemaakt met behulp van \texttt{prcomp(...)} en
\texttt{biplot(model)}.

    \begin{Verbatim}[commandchars=\\\{\}]
{\color{incolor}In [{\color{incolor}13}]:} biplot\PY{p}{(}prcomp\PY{p}{(}df\PY{p}{[}\PY{p}{,}\PY{l+m}{1}\PY{o}{:}\PY{l+m}{13}\PY{p}{]}\PY{p}{,} scale\PY{o}{=}\PY{n+nb+bp}{T}\PY{p}{)}\PY{p}{,} xlabs\PY{o}{=}df\PY{o}{\PYZdl{}}target\PY{p}{)}
\end{Verbatim}

    \begin{center}
    \adjustimage{max size={0.9\linewidth}{0.9\paperheight}}{output_41_0.png}
    \end{center}
    { \hspace*{\fill} \\}
    
    De bi-plot die door R is gegenereerd, komt goed overeen met de
zelfgemaakte bi-plot.

    \hypertarget{scree-plot-en-grenswaarden}{%
    	\newpage
\section{Scree plot en grenswaarden}\label{scree-plot-en-grenswaarden}}

    Met een scree plot wordt er bepaald hoeveel PCs er worden meegenomen in
een model. De grenswaarde die hiervoor geldt is als volgt bepaald:

\[ grens = \dfrac{\textrm{trace}(\textrm{Covariantie matrix van }A)}{v}\]

waarbij \(v\) het aantal variabelen is. Met de onderstaande code wordt
een scree plot gegenereerd:

    \begin{Verbatim}[commandchars=\\\{\}]
{\color{incolor}In [{\color{incolor}14}]:} \PY{c+c1}{\PYZsh{}eigenwaarden bepalen}
         lambdas \PY{o}{=} \PY{k+kp}{eigen}\PY{p}{(}cov\PY{p}{(}Xcs\PY{p}{)}\PY{p}{)}\PY{o}{\PYZdl{}}values
         lambdas \PY{o}{=} \PY{k+kp}{sort}\PY{p}{(}lambdas\PY{p}{,} decreasing\PY{o}{=}\PY{k+kc}{TRUE}\PY{p}{)}
         
         \PY{c+c1}{\PYZsh{} staafdiagram maken}
         plot\PY{p}{(}lambdas\PY{p}{,} type\PY{o}{=}\PY{l+s}{\PYZsq{}}\PY{l+s}{o\PYZsq{}}\PY{p}{,} pch\PY{o}{=}\PY{l+m}{18}\PY{p}{,} col\PY{o}{=}\PY{l+s}{\PYZsq{}}\PY{l+s}{red\PYZsq{}}\PY{p}{,} cex\PY{o}{=}\PY{l+m}{3}\PY{p}{,} lwd\PY{o}{=}\PY{l+m}{2}
              \PY{p}{,} main\PY{o}{=}\PY{l+s}{\PYZdq{}}\PY{l+s}{Scree plot\PYZdq{}}\PY{p}{,} xlab\PY{o}{=}\PY{l+s}{\PYZdq{}}\PY{l+s}{Index\PYZdq{}}\PY{p}{,} ylab\PY{o}{=}\PY{l+s}{\PYZdq{}}\PY{l+s}{Eigenvalue\PYZdq{}}\PY{p}{,}
             xlim\PY{o}{=}\PY{k+kt}{c}\PY{p}{(}\PY{l+m}{0}\PY{p}{,} \PY{k+kp}{length}\PY{p}{(}lambdas\PY{p}{)}\PY{p}{)}\PY{p}{)}
         
         \PY{c+c1}{\PYZsh{} grenswaarde bepalen}
         lim \PY{o}{=} \PY{k+kp}{trace}\PY{p}{(}\PY{p}{(}cov\PY{p}{(}Xcs\PY{p}{)}\PY{p}{)}\PY{p}{)} \PY{o}{/} \PY{k+kp}{length}\PY{p}{(}lambdas\PY{p}{)}
         
         \PY{c+c1}{\PYZsh{} grenswaarde tekenen}
         abline\PY{p}{(}h\PY{o}{=}lim\PY{p}{,} lt\PY{o}{=}\PY{l+s}{\PYZsq{}}\PY{l+s}{dashed\PYZsq{}}\PY{p}{,} lwd\PY{o}{=}\PY{l+m}{4}\PY{p}{,} col\PY{o}{=}\PY{l+s}{\PYZsq{}}\PY{l+s}{gray\PYZsq{}}\PY{p}{)}
         
         \PY{c+c1}{\PYZsh{} eigenwaarden in de grafiek plotten}
         text\PY{p}{(}lambdas \PY{o}{\PYZhy{}} \PY{l+m}{0.01}\PY{p}{,} labels\PY{o}{=}\PY{k+kp}{round}\PY{p}{(}lambdas\PY{p}{,}\PY{l+m}{2}\PY{p}{)}\PY{p}{)}
         grid\PY{p}{(}\PY{p}{)}
\end{Verbatim}

    \begin{center}
    \adjustimage{max size={0.9\linewidth}{0.9\paperheight}}{output_45_0.png}
    \end{center}
    { \hspace*{\fill} \\}
    
    Hier is te zien, omdat er is gestandardiseerd, dat de grenswaarde \(1\)
is. Dit betekend dat er \(5\) PCs zijn waarvan de eigenwaarde groter is
dan \(1\), en worden alleen deze meegenomen.

    Met R kan ook een scree plot snel worden gecontroleerd:

    \begin{Verbatim}[commandchars=\\\{\}]
{\color{incolor}In [{\color{incolor}15}]:} screeplot\PY{p}{(}prcomp\PY{p}{(}df\PY{p}{[}\PY{p}{,}\PY{l+m}{1}\PY{o}{:}\PY{l+m}{13}\PY{p}{]}\PY{p}{,} scale\PY{o}{=}\PY{n+nb+bp}{T}\PY{p}{)}\PY{p}{)}
         grid\PY{p}{(}\PY{p}{)}
\end{Verbatim}

    \begin{center}
    \adjustimage{max size={0.9\linewidth}{0.9\paperheight}}{output_48_0.png}
    \end{center}
    { \hspace*{\fill} \\}
    
    Met een visuele inspectie valt te zien dat de waarden overeen komen met
de zelfgemaakte scree plot.

    \hypertarget{errormatrices-berekenen}{%
\section{Errormatrices berekenen}\label{errormatrices-berekenen}}

    De waarnemingen kunnen we ook opschrijven in een model als lineaire
combinaties van de PCs en de t-scores. Het model hiervoor is:

\[ X_{cs} = \hat{X}_{cs} + E, \]

waarbij \(\hat{X}_{cs}\) de verklaarde variantie bevat en \(E\) de
onverklaarde variantie bevat. Voor dit model worden er \(5\) PCs
gebruikt, dus het gehele model is als volgt:

\[ X_{cs} = t_1 \cdot p^T_1 + t_2 \cdot p^T_2 + \cdots + t_5 \cdot p^T_5 + E_5. \]

    Omdat alle \(t\) scores al zijn berekend, is het bepalen van de error
matrices nog eenvoudiger. De formule hiervoor is
\(E_n = E_{n-1} - t_n \cdot p^T_n\). Vanuit de scree plot is er bepaald
dat de eerste vijf PCs worden meegenomen in het model. Met de
onderstaande code, worden de error matrices berekend.

    \begin{Verbatim}[commandchars=\\\{\}]
{\color{incolor}In [{\color{incolor}16}]:} E0 \PY{o}{=} Xcs
         E1 \PY{o}{=} E0 \PY{o}{\PYZhy{}} t1    \PY{o}{\PYZpc{}*\PYZpc{}} \PY{k+kp}{t}\PY{p}{(}p\PY{p}{[}\PY{p}{,}\PY{l+m}{1}\PY{p}{]}\PY{p}{)}
         E2 \PY{o}{=} E1 \PY{o}{\PYZhy{}} \PY{k+kp}{t}\PY{p}{[}\PY{p}{,}\PY{l+m}{2}\PY{p}{]} \PY{o}{\PYZpc{}*\PYZpc{}} \PY{k+kp}{t}\PY{p}{(}p\PY{p}{[}\PY{p}{,}\PY{l+m}{2}\PY{p}{]}\PY{p}{)}
         E3 \PY{o}{=} E2 \PY{o}{\PYZhy{}} \PY{k+kp}{t}\PY{p}{[}\PY{p}{,}\PY{l+m}{3}\PY{p}{]} \PY{o}{\PYZpc{}*\PYZpc{}} \PY{k+kp}{t}\PY{p}{(}p\PY{p}{[}\PY{p}{,}\PY{l+m}{3}\PY{p}{]}\PY{p}{)}
         E4 \PY{o}{=} E3 \PY{o}{\PYZhy{}} \PY{k+kp}{t}\PY{p}{[}\PY{p}{,}\PY{l+m}{4}\PY{p}{]} \PY{o}{\PYZpc{}*\PYZpc{}} \PY{k+kp}{t}\PY{p}{(}p\PY{p}{[}\PY{p}{,}\PY{l+m}{4}\PY{p}{]}\PY{p}{)}
         E5 \PY{o}{=} E4 \PY{o}{\PYZhy{}} \PY{k+kp}{t}\PY{p}{[}\PY{p}{,}\PY{l+m}{5}\PY{p}{]} \PY{o}{\PYZpc{}*\PYZpc{}} \PY{k+kp}{t}\PY{p}{(}p\PY{p}{[}\PY{p}{,}\PY{l+m}{5}\PY{p}{]}\PY{p}{)}
         E \PY{o}{=} \PY{k+kt}{list}\PY{p}{(}E0\PY{p}{,} E1\PY{p}{,} E2\PY{p}{,} E3\PY{p}{,} E4\PY{p}{,} E5\PY{p}{)}
\end{Verbatim}

    \hypertarget{mse-berekenen-en-variantie-analyse}{%
\section{MSE berekenen en
variantie-analyse}\label{mse-berekenen-en-variantie-analyse}}

    Om te bepalen hoeveel procent van de variante wordt verklaard per
principeel component, wordt er een variantie-analyse uitgevoerd.
Hiervoor worden voor alle error matrices \(E_1, \ldots, E_5\) de
variantie bepaald.

Met de volgende formule wordt de variantie voor een matrix bepaald:

\[ \textrm{matrix.var}(\mathbf{E}) = \dfrac{\sum\limits_{i=1}^N \sum\limits_{j=1}^v e^2_{ij}}{N\cdot v} = \dfrac{\textrm{trace}(E^T\cdot E)}{N\cdot v}. \]

Vervolgens wordt dit gebruik om een tabel op te stellen met de volgende
kolommen:

\begin{itemize}
\tightlist
\item
  Aantal PCs in het model
\item
  Totale variantie
\item
  Resterende variantie in \%
\item
  Verklaarde variantie in \%
\item
  Verklaarde variantie per PC
\end{itemize}

Met de onderstaande code wordt deze tabel opgesteld.

    \begin{Verbatim}[commandchars=\\\{\}]
{\color{incolor}In [{\color{incolor}17}]:} \PY{c+c1}{\PYZsh{} aantal PCs}
         pcs \PY{o}{=} \PY{l+m}{0}\PY{o}{:}\PY{l+m}{5}
         
         \PY{c+c1}{\PYZsh{} totale variantie bepalen}
         total.var \PY{o}{=} \PY{k+kt}{c}\PY{p}{(}matrix.var\PY{p}{(}E0\PY{p}{)}\PY{p}{,} matrix.var\PY{p}{(}E1\PY{p}{)}\PY{p}{,} matrix.var\PY{p}{(}E2\PY{p}{)}\PY{p}{,}
                      matrix.var\PY{p}{(}E3\PY{p}{)}\PY{p}{,} matrix.var\PY{p}{(}E4\PY{p}{)}\PY{p}{,} matrix.var\PY{p}{(}E5\PY{p}{)}\PY{p}{)}
         
         \PY{c+c1}{\PYZsh{} resterende variantie bepalen}
         rest.var \PY{o}{=} \PY{k+kt}{c}\PY{p}{(}\PY{l+m}{1}\PY{p}{,} total.var\PY{p}{[}\PY{l+m}{2}\PY{p}{]} \PY{o}{/} total.var\PY{p}{[}\PY{l+m}{1}\PY{p}{]}\PY{p}{,} total.var\PY{p}{[}\PY{l+m}{3}\PY{p}{]} \PY{o}{/} total.var\PY{p}{[}\PY{l+m}{1}\PY{p}{]}\PY{p}{,} 
                      total.var\PY{p}{[}\PY{l+m}{4}\PY{p}{]} \PY{o}{/} total.var\PY{p}{[}\PY{l+m}{1}\PY{p}{]}\PY{p}{,} total.var\PY{p}{[}\PY{l+m}{5}\PY{p}{]} \PY{o}{/} total.var\PY{p}{[}\PY{l+m}{1}\PY{p}{]}\PY{p}{,} total.var\PY{p}{[}\PY{l+m}{6}\PY{p}{]}\PY{p}{)}
         
         \PY{c+c1}{\PYZsh{} verklaarde variantie bepalen}
         expl.var \PY{o}{=} \PY{l+m}{1} \PY{o}{\PYZhy{}} rest.var
         
         \PY{c+c1}{\PYZsh{} verklaarde variantie per PC bepalen}
         var.per.pc \PY{o}{=} \PY{k+kt}{c}\PY{p}{(}\PY{l+m}{0}\PY{p}{,} \PY{k+kp}{diff}\PY{p}{(}expl.var\PY{p}{)}\PY{p}{)}
         
         \PY{c+c1}{\PYZsh{} tabel opstellen}
         df.vars \PY{o}{=} \PY{k+kt}{data.frame}\PY{p}{(}pcs\PY{p}{,} total.var\PY{p}{,} rest.var\PY{p}{,} expl.var\PY{p}{,} var.per.pc\PY{p}{)}
         \PY{k+kp}{colnames}\PY{p}{(}df.vars\PY{p}{)} \PY{o}{=} \PY{k+kt}{c}\PY{p}{(}\PY{l+s}{\PYZdq{}}\PY{l+s}{PCs in model\PYZdq{}}\PY{p}{,} \PY{l+s}{\PYZdq{}}\PY{l+s}{Total variance\PYZdq{}}\PY{p}{,} \PY{l+s}{\PYZdq{}}\PY{l+s}{Rest variance in \PYZpc{}\PYZdq{}}\PY{p}{,} 
                               \PY{l+s}{\PYZdq{}}\PY{l+s}{Explained variance in \PYZpc{}\PYZdq{}}\PY{p}{,} \PY{l+s}{\PYZdq{}}\PY{l+s}{Explained variance per PC\PYZdq{}}\PY{p}{)}
         df.vars
\end{Verbatim}

    \begin{tabular}{r|lllll}
 PCs in model & Total variance & Rest variance in \% & Explained variance in \% & Explained variance per PC\\
\hline
	 0          & 0.9966997  & 1.0000000  & 0.0000000  & 0.00000000\\
	 1          & 0.7848606  & 0.7874595  & 0.2125405  & 0.21254053\\
	 2          & 0.6670436  & 0.6692524  & 0.3307476  & 0.11820708\\
	 3          & 0.5732899  & 0.5751882  & 0.4248118  & 0.09406418\\
	 4          & 0.4827324  & 0.4843309  & 0.5156691  & 0.09085735\\
	 5          & 0.4043791  & 0.4043791  & 0.5956209  & 0.07995181\\
\end{tabular}


    
    Wat opvalt is de \(PC_1\) een redelijk deel van de variantie verklaard,
namelijk \(21\%\). De volgende, \(PC_2\), voegt slechts \(12\%\) toe.

Ook is de tabel te controleren met R. Dezelfde kan tabel kan worden
gevonden met \texttt{summary(...)}.

    \begin{Verbatim}[commandchars=\\\{\}]
{\color{incolor}In [{\color{incolor}18}]:} \PY{k+kp}{summary}\PY{p}{(}prcomp\PY{p}{(}df\PY{p}{[}\PY{p}{,}\PY{l+m}{1}\PY{o}{:}\PY{l+m}{13}\PY{p}{]}\PY{p}{,} scale\PY{o}{=}\PY{n+nb+bp}{T}\PY{p}{)}\PY{p}{)}
\end{Verbatim}

    
    \begin{verbatim}
Importance of components:
                          PC1    PC2     PC3     PC4     PC5     PC6     PC7
Standard deviation     1.6622 1.2396 1.10582 1.08681 1.01092 0.98489 0.92885
Proportion of Variance 0.2125 0.1182 0.09406 0.09086 0.07861 0.07462 0.06637
Cumulative Proportion  0.2125 0.3307 0.42481 0.51567 0.59428 0.66890 0.73527
                           PC8    PC9    PC10    PC11    PC12   PC13
Standard deviation     0.88088 0.8479 0.78840 0.72808 0.65049 0.6098
Proportion of Variance 0.05969 0.0553 0.04781 0.04078 0.03255 0.0286
Cumulative Proportion  0.79495 0.8503 0.89807 0.93885 0.97140 1.0000
    \end{verbatim}

    
    Hier valt te zien dat de verklaarde variantie in \% voor een model met
één PC overeenkomt met \(0.212\). De tabel kan ook worden weergegeven in
een grafiek. Dit wordt gedaan met de onderstaande code.

    \begin{Verbatim}[commandchars=\\\{\}]
{\color{incolor}In [{\color{incolor}25}]:} \PY{c+c1}{\PYZsh{} lijngrafiek maken voor verklaarde variantie}
         plot\PY{p}{(}df.vars\PY{p}{[}\PY{p}{,}\PY{l+m}{1}\PY{p}{]}\PY{p}{,} df.vars\PY{p}{[}\PY{p}{,}\PY{l+m}{3}\PY{p}{]}\PY{p}{,} type\PY{o}{=}\PY{l+s}{\PYZsq{}}\PY{l+s}{o\PYZsq{}}\PY{p}{,} col\PY{o}{=}\PY{l+s}{\PYZsq{}}\PY{l+s}{red\PYZsq{}}\PY{p}{,} lwd\PY{o}{=}\PY{l+m}{2}\PY{p}{,} pch\PY{o}{=}\PY{l+m}{18}\PY{p}{,} cex\PY{o}{=}\PY{l+m}{2}\PY{p}{,} 
              main\PY{o}{=}\PY{l+s}{\PYZdq{}}\PY{l+s}{Explained variance per PC\PYZdq{}}\PY{p}{,} ylim\PY{o}{=}\PY{k+kt}{c}\PY{p}{(}\PY{l+m}{0}\PY{p}{,}\PY{l+m}{1}\PY{p}{)}\PY{p}{,}
              ylab\PY{o}{=}\PY{l+s}{\PYZdq{}}\PY{l+s}{\PYZpc{} of Total variance\PYZdq{}}\PY{p}{,} xlab\PY{o}{=}\PY{l+s}{\PYZdq{}}\PY{l+s}{\PYZsh{} of PCs\PYZdq{}}\PY{p}{)}
         
         \PY{c+c1}{\PYZsh{} lijngrafiek toevoegen voor onverklaarde variantie}
         lines\PY{p}{(}df.vars\PY{p}{[}\PY{p}{,}\PY{l+m}{1}\PY{p}{]}\PY{p}{,} df.vars\PY{p}{[}\PY{p}{,}\PY{l+m}{4}\PY{p}{]}\PY{p}{,} type\PY{o}{=}\PY{l+s}{\PYZsq{}}\PY{l+s}{o\PYZsq{}}\PY{p}{,} col\PY{o}{=}\PY{l+s}{\PYZsq{}}\PY{l+s}{gold\PYZsq{}}\PY{p}{,} pch\PY{o}{=}\PY{l+m}{17}\PY{p}{,} lwd\PY{o}{=}\PY{l+m}{2}\PY{p}{,} cex\PY{o}{=}\PY{l+m}{2}\PY{p}{)}
         
         legend\PY{p}{(}\PY{l+s}{\PYZdq{}}\PY{l+s}{bottomright\PYZdq{}}\PY{p}{,} col\PY{o}{=}\PY{k+kt}{c}\PY{p}{(}\PY{l+s}{\PYZsq{}}\PY{l+s}{red\PYZsq{}}\PY{p}{,} \PY{l+s}{\PYZsq{}}\PY{l+s}{gold\PYZsq{}}\PY{p}{)}\PY{p}{,} 
                legend\PY{o}{=}\PY{k+kt}{c}\PY{p}{(}\PY{l+s}{\PYZsq{}}\PY{l+s}{Remaining variance\PYZsq{}}\PY{p}{,} \PY{l+s}{\PYZsq{}}\PY{l+s}{Explained variance\PYZsq{}}\PY{p}{)}\PY{p}{,} pch\PY{o}{=}\PY{k+kt}{c}\PY{p}{(}\PY{l+m}{18}\PY{p}{,} \PY{l+m}{17}\PY{p}{)}\PY{p}{)}
         grid\PY{p}{(}\PY{p}{)}
\end{Verbatim}

    \begin{center}
    \adjustimage{max size={0.9\linewidth}{0.9\paperheight}}{output_60_0.png}
    \end{center}
    { \hspace*{\fill} \\}
    
    Wat opvalt is dat bij de grenswaarde, waardoor er \(5\) PCs werden
opgenomen, de lijnen elkaar kruisen. Op het moment dat er \(5\) PCs in
het model zijn opgenomen, is de verklaarde variantie \(59\%\).

    Een andere aspect is het kijken naar mogelijke uitbijters. Hiervoor
wordt er gekeken naar de gemiddelde kolom of rij varianties van de error
matrices. Omdat er nogal veel waarnemingen zijn, is een grafiek van de
rijen niet overzichtelijk. Echter is het wel mogelijk om van de kolommen
een error matrix plot te maken. Met de onderstaande code wordt deze
grafiek gegenereerd.

    \begin{Verbatim}[commandchars=\\\{\}]
{\color{incolor}In [{\color{incolor}20}]:} \PY{k+kp}{rownames}\PY{p}{(}E5\PY{p}{)} \PY{o}{=} \PY{l+m}{1}\PY{o}{:}\PY{k+kp}{nrow}\PY{p}{(}df\PY{p}{)}
         barplot\PY{p}{(}\PY{k+kp}{colMeans}\PY{p}{(}E5\PY{o}{\PYZca{}}\PY{l+m}{2}\PY{p}{)}\PY{p}{,} las\PY{o}{=}\PY{l+m}{2}\PY{p}{,} col\PY{o}{=}\PY{l+s}{\PYZsq{}}\PY{l+s}{orangered1\PYZsq{}}\PY{p}{)}
         grid\PY{p}{(}\PY{p}{)}
         title\PY{p}{(}main\PY{o}{=}\PY{l+s}{\PYZdq{}}\PY{l+s}{Final Error Matrix (Variables)\PYZdq{}}\PY{p}{)}
\end{Verbatim}

    \begin{center}
    \adjustimage{max size={0.9\linewidth}{0.9\paperheight}}{output_63_0.png}
    \end{center}
    { \hspace*{\fill} \\}
    
    Hier is te zien dat de fout over alle variabelen goed verdeeld is. Er is
geen variabele die een grote fout veroorzaakt in de laatste error matrix
(\(E_5\)). Vanuit dit oogpunt zitten er geen uitbijters in de gegevens.

    \hypertarget{samplevarianties-en-variabele-varianties}{%
\section{Samplevarianties en variabele
varianties}\label{samplevarianties-en-variabele-varianties}}

    Een andere grafiek laat zien hoe de variantie afneemt bij het toevoegen
van meerdere PCs. Ook hier is er de mogelijkheid om dit voor de kolommen
en de rijen te doen. Aangezien er veel rijen zijn, wordt dit alleen voor
de kolommen gedaan. De grafiek die anders ontstaat is niet duidelijk.

Met de onderstaande code wordt deze grafiek gegenereerd voor de
variabelen.
\newpage
    \begin{Verbatim}[commandchars=\\\{\}]
{\color{incolor}In [{\color{incolor}21}]:} \PY{c+c1}{\PYZsh{} kolomgemiddelden bepalen voor de error matrices}
         variable.vars \PY{o}{=} \PY{k+kt}{data.frame}\PY{p}{(}\PY{k+kp}{colMeans}\PY{p}{(}\PY{p}{(}E0\PY{p}{)}\PY{o}{\PYZca{}}\PY{l+m}{2}\PY{p}{)}\PY{p}{,}\PY{k+kp}{colMeans}\PY{p}{(}\PY{p}{(}E1\PY{p}{)}\PY{o}{\PYZca{}}\PY{l+m}{2}\PY{p}{)}\PY{p}{,}
                                    \PY{k+kp}{colMeans}\PY{p}{(}\PY{p}{(}E2\PY{p}{)}\PY{o}{\PYZca{}}\PY{l+m}{2}\PY{p}{)}\PY{p}{,}\PY{k+kp}{colMeans}\PY{p}{(}\PY{p}{(}E3\PY{p}{)}\PY{o}{\PYZca{}}\PY{l+m}{2}\PY{p}{)}\PY{p}{,}
                                    \PY{k+kp}{colMeans}\PY{p}{(}\PY{p}{(}E4\PY{p}{)}\PY{o}{\PYZca{}}\PY{l+m}{2}\PY{p}{)}\PY{p}{,}\PY{k+kp}{colMeans}\PY{p}{(}\PY{p}{(}E5\PY{p}{)}\PY{o}{\PYZca{}}\PY{l+m}{2}\PY{p}{)}\PY{p}{)}
         
         \PY{c+c1}{\PYZsh{} kleuren}
         colors \PY{o}{=} \PY{k+kt}{c}\PY{p}{(}\PY{l+s}{\PYZsq{}}\PY{l+s}{\PYZsh{}018786\PYZsq{}}\PY{p}{,} \PY{l+s}{\PYZsq{}}\PY{l+s}{\PYZsh{}019592\PYZsq{}}\PY{p}{,} \PY{l+s}{\PYZsq{}}\PY{l+s}{\PYZsh{}01a299\PYZsq{}}\PY{p}{,} 
                    \PY{l+s}{\PYZsq{}}\PY{l+s}{\PYZsh{}00b3a6\PYZsq{}}\PY{p}{,} \PY{l+s}{\PYZsq{}}\PY{l+s}{\PYZsh{}00c4b4\PYZsq{}}\PY{p}{,} \PY{l+s}{\PYZsq{}}\PY{l+s}{\PYZsh{}70efde\PYZsq{}}\PY{p}{)}
         
         \PY{c+c1}{\PYZsh{} staafdiagram maken}
         barplot\PY{p}{(}\PY{k+kp}{t}\PY{p}{(}variable.vars\PY{p}{)}\PY{p}{,} las\PY{o}{=}\PY{l+m}{2}\PY{p}{,} beside\PY{o}{=}\PY{n+nb+bp}{T}\PY{p}{,} 
                 main\PY{o}{=}\PY{l+s}{\PYZdq{}}\PY{l+s}{Variable variances plot\PYZdq{}}\PY{p}{,} col\PY{o}{=}colors\PY{p}{)}
         legend\PY{p}{(}\PY{l+s}{\PYZdq{}}\PY{l+s}{topright\PYZdq{}}\PY{p}{,} legend\PY{o}{=}\PY{k+kt}{c}\PY{p}{(}\PY{l+s}{\PYZdq{}}\PY{l+s}{PC 0\PYZdq{}}\PY{p}{,} \PY{l+s}{\PYZdq{}}\PY{l+s}{PC 1\PYZdq{}}\PY{p}{,} \PY{l+s}{\PYZdq{}}\PY{l+s}{PC 2\PYZdq{}}\PY{p}{,} 
                                     \PY{l+s}{\PYZdq{}}\PY{l+s}{PC 3\PYZdq{}}\PY{p}{,} \PY{l+s}{\PYZdq{}}\PY{l+s}{PC 4\PYZdq{}}\PY{p}{,} \PY{l+s}{\PYZdq{}}\PY{l+s}{PC 5\PYZdq{}}\PY{p}{)}\PY{p}{,} fill\PY{o}{=}colors\PY{p}{)}
         grid\PY{p}{(}\PY{p}{)}
\end{Verbatim}

    \begin{center}
    \adjustimage{max size={0.9\linewidth}{0.9\paperheight}}{output_67_0.png}
    \end{center}
    { \hspace*{\fill} \\}
    
    Hier valt te zien dat een gedeelte van de variantie is opgenomen in het
eerste principele component \(PC_1\). Voornamelijk voor de variabele met
een grote loading in \(p_1\), is er te zien dat een deel van de
variantie wordt verklaard met \(PC_1\). Doordat de variabelen
\texttt{thalach}, \texttt{exang}, \texttt{oldpeak} en \texttt{slope} een
grote loading hebben, hebben ze vermoedelijk een grote invloed op het
feit of iemand een hartziekte heeft. De analyse laat zien dat deze
variabelen het zwaarst mee wegen in het eerste model.

Vanuit de loading plot is ook goed te zien dat juist deze variabelen ook
de correcte richting uitoefenen op waar de waarnemingen belanden in de
score plot.

    \hypertarget{verschillen-en-verbanden}{%
\section{Verschillen en verbanden}\label{verschillen-en-verbanden}}

    In de vorige paragraaf is gebleken dat een gedeelte van de variantie in
het model met \(1\) PC wordt verklaard door de variabelen
\texttt{thalach}, \texttt{exang}, \texttt{oldpeak} en \texttt{slope}. Om
te kijken of er inderdaad een significant verschil is tussen deze
variabelen en de splitsingsvariabele \texttt{target}, is er met SPSS een
t-toets uitgevoerd voor het verschil in gemiddelden. Hiervoor is alleen
van \texttt{slope} geen toets uitgevoerd, omdat in dit geval beide
variabelen nominaal zijn.

Het resultaat van de toets tussen alle ratio variabelen en de
splitsingsvariabele \texttt{target} is te vinden in de onderstaande
tabel:


 \begin{center}
	\adjustimage{max size={0.7\linewidth}{0.7\paperheight}}{t-toets.PNG}
\end{center}

    Hier is te zien dat \texttt{age}, \texttt{trestbps}, \texttt{thalach} en
\texttt{oldpeak} allen een significant verschil vertonen. In de loading
plot is te zien dat \texttt{trestbps} en \texttt{age} beide een richting
uitoefenen die bijna loodrecht op \texttt{thalach} en \texttt{oldpeak}
staat. Wat ook opvalt is dat voor de groep met een hartziekte, de
maximaal behaalde hartslag tijden de oefening, gemiddeld bijna \(20\)
BPM hoger is dan bij de groep zonder hartziekte. Ditzelfde geldt voor
\texttt{oldpeak}, deze scoort gemiddeld \texttt{1} lager bij de groep
met een hartziekte. Dit geeft de volgende vraag, is er een verband
tussen \texttt{oldpeak} en \texttt{thalach}? Vanuit het onderzoek met
meervoudige regressie is gebleken dat de correlatiecoëfficiënt
hiertussen \(-0.344\) is. Een snelle toets in SPSS toont aan dat dit
resultaat significant is met een p-waarde van \(0.00\) (toetsresultaat
achterwege gelaten). Er is dus een negatief zwak verband tussen
\texttt{oldpeak} en \texttt{thalach}.

    \hypertarget{conclusie}{%
\section{Conclusie}\label{conclusie}}

    Vanuit de principele componenten analyse en de verschiltoets is gebleken
dat de variabelen die de grootste invloed uitoefenen op de mogelijkheid
of iemand een hartziekte heeft, \texttt{oldpeak} en \texttt{thalach}
zijn. Uit de loading plot valt te zien dat \texttt{slope} en
\texttt{exang} een grote invloed uitoefenen op de mogelijkheid of iemand
een hartziekte heeft. Echter zijn dit nominale variabelen en is hiervan
geen verschiltoets uitgevoerd.

De variabele \texttt{oldpeak} staat voor \emph{ST depression induced by
exercise relative to test}. Hiermee wordt een specifiek fenomeen bedoeld
wat voorkomt in een ECG (electrocardiograph) wat een indicatie is voor
een onderliggend probleem genaamd ischemia. {[}1{]} Als een van de
bloedvatten verstoopt raakt, onstaat er een ernstig probleem genaamd
\emph{ischemia}. {[}4{]} De variabele \texttt{slope} duidt een ander
fenomeen aan in een ECG, wat ook duidt op een onderliggend hartprobleem.

De andere variabele \texttt{thalach} geeft de maximaal behaalde hartslag
tijdens de oefening aan. Hier valt het op dat bij de mensen met een
hartprobleem, deze gemiddeld \(20\) BPM hoger is.

Beide variabelen geven een goede indicatie op mogelijke hartproblemen.
Aangezien niet iedereen thuis een ECG heeft, kan je dit zelf niet
controleren. Echter is het wel mogelijk om zelf de maximaal behaalde
hartslag tijdens een oefening bij te houden. Als deze hoger ligt dan het
gemiddelde, wat bepaald is per leeftijd, is dat een goede reden voor een
doktersbezoek.

De laatste variabele, \texttt{exang}, geeft aan of iemand pijn of
oncomfortabelheden in zijn of haar borst heeft ervaren gedurende de
oefening. Dit fenomeen wordt angina genoemd. Dit word mogelijk
veroorzaakt doordat er te weinig zuurstof-rijk bloed bij die spier komt.
Ook dit is een symptoom van een onderliggend hartprobleem. {[}3{]}

    \hypertarget{referenties}{%
\section{Referenties}\label{referenties}}

    In dit rapport zijn de volgende referenties geraadpleegd:

\begin{enumerate}
\def\labelenumi{\arabic{enumi}.}
\tightlist
\item
  https://ecgwaves.com/ecg-topic/ecg-st-segment-depression-ischemia-infarction-differential-diagnoses/
\item
  https://www.webmd.com/heart-disease/electrocardiogram-ekgs\#1
\item
  https://www.heart.org/en/health-topics/heart-attack/angina-chest-pain
\item
  https://www.webmd.com/heart-disease/what-is-ischemia\#1
\end{enumerate}


    % Add a bibliography block to the postdoc
    
    
    
    \end{document}
